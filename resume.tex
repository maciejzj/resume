\documentclass{article}

\usepackage{geometry}
\usepackage{titlesec}
\usepackage[hidelinks]{hyperref}
\usepackage{url}
\usepackage{marginnote}
\usepackage{lastpage}
\usepackage{fancyhdr}
\usepackage{sectsty} \geometry{
	left = 3cm,
	top = 2cm,
	bottom = 2cm
}

\urlstyle{same}

\pagestyle{fancy}
\fancyhf{}
\renewcommand\headrulewidth{0pt}
\rfoot{\thepage\ of \pageref{LastPage}}
\reversemarginpar

\begin{document}
\hfill
% Contact info
\begin{minipage}{0.5\textwidth}\raggedleft
	{\Large \textbf{Maciej Ziaja}} \\
	Born 22.05.1997 \\
	Gliwicka 19 $\circ$ 44--160 Rudno \\
	+48 729 920 048 $\circ$ maciejzj@icloud.com \\
	maciejzj.xyz $\circ$ github.com/maciejzj
\end{minipage}

% Professional experience
\section*{Experience}

\hrule \medskip

\marginnote{\small \textit{Since October 2020}}
\textsc{KP Labs}, Gliwice \\
\textbf{Machine Learning Software Engineer} -- Development and prototyping of
deep neural networks for image processing with Python, deployment on embedded
Linux devices with C++.
\begin{itemize}
	\item Involved in creation of several deep learning image processing models
		for segmentation (roads, buildings, clouds, etc.) and regression tasks
		(arctic ice cover properties) in satellite imagery. Responsible for
		data management, preprocessing, training pipelines, model architecture
		(CNNs and Transformers), and optimization.
	\item Designed neural networks deployment and benchmarking processes for
		AI-capable state-of-the-art satellite mission Intuition-1. Enabled
		on-board neural network inference hardware acceleration utilizing
		\textit{Xilinx's Vitis AI} framework in Python and C++. The satellite
		platform equipped with the deep learning models currently operates
		in-orbit.
	\item Involved in \textit{super-resolution} tasks (single-image,
		multi-image, pansharpening, data fusion) applied to satellite imagery,
		such as data generation, dataset curation, and super-resolution deep
		neural networks training including Transformer-based and GAN models.
	\item Maintained in-house data processing and experimentation MLOps systems
		based on DVC, MLflow, Docker, and Jenkins.
	\item Involved in preparations and orations at various industry and
		scientific events. Published R\&D research results in several venues
		redacted by Nature, Springer, and IEEE.
\end{itemize}
\marginnote{\small \textit{July--October 2019}}
\textsc{Display Link}, Katowice \\
\textbf{Intern Development Engineer} -- Embedded C++ programming in real time
environment for video processing devices.

% Skills and tech stack
\section*{Skills}

\hrule \medskip

\begin{description}
	\item[Python programming for machine learning, application development and scripting]
		deep learning with \textit{Tensorflow} and \textit{PyTorch},
		machine learning with \textit{scikit-learn},
		data manipulation with \textit{NumPy} and \textit{Pandas} from SQL and NoSQL sources,
		data visualization with \textit{Matplotlib} and \textit{Dash \& Plotly} combo,
		Jupyter notebooks,
		computer vision with \textit{Scikit-image},
		data versioning with \textit{DVC},
		experiment management with \textit{MLflow},
		code quality with \textit{pytest}, \textit{flake8}, and \textit{mypy}.
	\item[C++ programming in modern standards]
		\textit{STL} familiarity,
		\textit{CMake} build tool,
		unit testing with \textit{Google Test} and \textit{Google Mock},
		\textit{Clang} toolchain familiarity (\textit{clang-tidy, clang-format, lldb}).
	\item[Embedded programming and C programming in Linux environment]
		\textit{STM32} microcontrollers,
		\textit{FreeRTOS} real time operating system,
		\textit{make} automation tool.
	\item[Developer tools]
		\textit{git} version control system,
		\textit{docker} containerization,
		familiarity with \textit{GitHub Actions, GitLab CI/CD} automation tools,
		familiarity with \textit{AWS} cloud computing services and
		\textit{architecture-as-code} with \textit{Terraform} and
		\textit{Ansible}, typesetting in \textit{Latex}, ability to work in
		\textit{Scrum}.
	\item[Linux operating system]
		administration and development,
		shell scripting,
		text processing with \textit{awk} and \textit{sed}.
	\item[Hardware design and multimedia processing]
		\textit{Autodesk Fusion360 and Eagle},
		\textit{Affinity} suite for image editing,
		\textit{DaVinci Resolve} and \textit{Final Cut Pro X} for video editing.
	\item[Languages]
		native Polish,
		proficient English,
		basic German.
\end{description}

% Academic education
\section*{Education}

\hrule \medskip

\marginnote{\small \textit{Since 2022}}
\noindent
\textsc{Silesian University of Technology} \\
\textbf{Doctor of Philisophy (PhD) student} -- \textit{Computer Science, Department of Algorithmics and Software.}

\medskip
\marginnote{\small \textit{2020--2021}}
\noindent
\textsc{Silesian University of Technology} \\
\textbf{Master of Engineering} -- \textit{Computer Science, System Software major.} \\
Thesis topic: \textit{Data augmentation for super-resolution reconstruction using deep convolutional neural networks}. \\
Graduated with distinction.

\medskip
\marginnote{\small \textit{2016--2020}}
\noindent
\textsc{Silesian University of Technology} \\
\textbf{Bachelor of Engineering} -- \textit{Automatic Control and Robotics, Information Technologies major.} \\
Thesis topic: \textit{Grains detection in thermal images with use of neural networks}. \\
Graduated with distinction.

% Scientific publications
\section*{Scientific publications}
\hrule \medskip
\begin{itemize}
	\item B. Grabowski et al., Squeezing adaptive deep learning methods with
		knowledge distillation for on-board cloud detection, Engineering
		Applications of Artificial Intelligence, Volume 132, 2024,
	\item P. Kowaleczko et al., A Real-World Benchmark for Sentinel-2
		Multi-Image Super-Resolution. Sci Data 10, 644 (2023).
	\item M. Ziaja et al Benchmarking Deep Learning for On-Board Space
		Applications. Remote Sens. 2021, 13, 3981
	% \item Krauze, P. et al. (2023). Identification of Magnetorheological Damper
	% 	Model for Off-Road Vehicle Suspension. In: Pawelczyk, M., Bismor, D.,
	% 	Ogonowski, S., Kacprzyk, J. (eds) Advanced, Contemporary Control. PCC
	% 	2023. Lecture Notes in Networks and Systems, vol 708. Springer, Cham.
	\item M. Ziaja et al., "Hyperspectral Image Pansharpening: The Prisma Case
		Study," IGARSS 2023 - 2023 IEEE International Geoscience and Remote
		Sensing Symposium, Pasadena, CA, USA, 2023, pp. 1633-1636.
	\item M. Kawulok et al., "Understanding the Value of Hyperspectral Image
		Super-Resolution from Prisma Data," IGARSS 2023 - 2023 IEEE
		International Geoscience and Remote Sensing Symposium, Pasadena, CA,
		USA, 2023, pp. 1489-1492.
	\item B. Grabowski et al., "Are Cloud Detection U-Nets Robust Against
		in-Orbit Image Acquisition Conditions?," IGARSS 2022 - 2022 IEEE
		International Geoscience and Remote Sensing Symposium, Kuala Lumpur,
		Malaysia, 2022, pp. 239-242.
	\item B. Grabowski, M. Ziaja, M. Kawulok and J. Nalepa, "Towards Robust
		Cloud Detection in Satellite Images Using U-Nets," 2021 IEEE
		International Geoscience and Remote Sensing Symposium IGARSS, Brussels,
		Belgium, 2021, pp. 4099-4102.
\end{itemize}
	
\section*{Personal projects}
\hrule \medskip
\begin{itemize}
	\item \textit{IT Jobs Meta} -- Data pipeline and meta-analysis
		dashboard for IT job postings from \textit{No Fluff Jobs} website.
		Features data scraping, cleaning, analysis, and interactive dashboard.
		Implemented in \textit{Python} and \textit{Pandas}, deployed with
		\textit{Ansible} and \textit{Terraform} to \textit{AWS}, available
		online at itjobsmeta.net.
	\item Other small software on my \href{https://github.com/maciejzj}{GitHub}:
		Linux dotfiles, mobile robot, shell-based blogging engine, high-altitude
		balloon embedded software, and more.
\end{itemize}

\section*{Certficicates}
\hrule \medskip
\begin{itemize}
	\item October 2021 -- \textit{AWS Cloud Technical Essentials} authorized by Amazon Web
	Services, offered through Coursera platform.
\end{itemize}

\vfill
\tiny{
\textit{I agree to the processing of personal data provided in this document for
realising the recruitment process pursuant to the Personal Data Protection Act
of 10 May 2018 (Journal of Laws 2018, item 1000) and in agreement with
Regulation (EU) 2016/679 of the European Parliament and of the Council of 27
April 2016 on the protection of natural persons with regard to the processing of
personal data and on the free movement of such data, and repealing Directive
95/46/EC (General Data Protection Regulation).}
}
\end{document}
